% THIS IS SIGPROC-SP.TEX - VERSION 3.1
% WORKS WITH V3.2SP OF ACM_PROC_ARTICLE-SP.CLS
% APRIL 2009
%
% It is an example file showing how to use the 'acm_proc_article-sp.cls' V3.2SP
% LaTeX2e document class file for Conference Proceedings submissions.
% ----------------------------------------------------------------------------------------------------------------
% This .tex file (and associated .cls V3.2SP) *DOES NOT* produce:
%       1) The Permission Statement
%       2) The Conference (location) Info information
%       3) The Copyright Line with ACM data
%       4) Page numbering
% ---------------------------------------------------------------------------------------------------------------
% It is an example which *does* use the .bib file (from which the .bbl file
% is produced).
% REMEMBER HOWEVER: After having produced the .bbl file,
% and prior to final submission,
% you need to 'insert'  your .bbl file into your source .tex file so as to provide
% ONE 'self-contained' source file.
%
% Questions regarding SIGS should be sent to
% Adrienne Griscti ---> griscti@acm.org
%
% Questions/suggestions regarding the guidelines, .tex and .cls files, etc. to
% Gerald Murray ---> murray@hq.acm.org
%
% For tracking purposes - this is V3.1SP - APRIL 2009

\documentclass{acm_proc_article-sp}
\usepackage[none]{hyphenat}
\begin{document}

\title{Detecting and Combating ARP Spoofing}
%
% You need the command \numberofauthors to handle the 'placement
% and alignment' of the authors beneath the title.
%
% For aesthetic reasons, we recommend 'three authors at a time'
% i.e. three 'name/affiliation blocks' be placed beneath the title.
%
% NOTE: You are NOT restricted in how many 'rows' of
% "name/affiliations" may appear. We just ask that you restrict
% the number of 'columns' to three.
%
% Because of the available 'opening page real-estate'
% we ask you to refrain from putting more than six authors
% (two rows with three columns) beneath the article title.
% More than six makes the first-page appear very cluttered indeed.
%
% Use the \alignauthor commands to handle the names
% and affiliations for an 'aesthetic maximum' of six authors.
% Add names, affiliations, addresses for
% the seventh etc. author(s) as the argument for the
% \additionalauthors command.
% These 'additional authors' will be output/set for you
% without further effort on your part as the last section in
% the body of your article BEFORE References or any Appendices.

\numberofauthors{4} %  in this sample file, there are a *total*
% of EIGHT authors. SIX appear on the 'first-page' (for formatting
% reasons) and the remaining two appear in the \additionalauthors section.
%
\author{
% You can go ahead and credit any number of authors here,
% e.g. one 'row of three' or two rows (consisting of one row of three
% and a second row of one, two or three).
%
% The command \alignauthor (no curly braces needed) should
% precede each author name, affiliation/snail-mail address and
% e-mail address. Additionally, tag each line of
% affiliation/address with \affaddr, and tag the
% e-mail address with \email.
%
% 1st. author
\alignauthor
Chai Ming Xuan\\
       \affaddr{National University of Singapore}\\
       \email{mingxuan@u.nus.edu}
% 2nd. author
\alignauthor
Er Xue Hui\\
       \affaddr{National University of Singapore}\\
       \email{xuehuier@u.nus.edu}
% 3rd. author
\alignauthor 
Wu Wenqi\\
       \affaddr{National University of Singapore}\\
       \email{wenqi.wu@u.nus.edu}
\and  % use '\and' if you need 'another row' of author names
% 4th. author
\alignauthor Zhu Chunqi\\
       \affaddr{National University of Singapore}\\
       \email{chunqi@u.nus.edu}
}
\maketitle
\begin{abstract}
Our project aims to create a program for users to actively detect if they are victims of ARP spoofing, and to offer ways to protect themselves.
\end{abstract}

% A category with the (minimum) three required fields
\category{C.2.0}{Computer-Communication Networks}{General - Data Communications, Security and Protection}

%A category including the fourth, optional field follows...
\category{D.4.6}{Security and Protection}{Authentication,\\ Verification}

\terms{Network Security}

\keywords{arp spoofing, network security} % NOT required for Proceedings

\section{Introduction}
The Address Resolution Protocol (ARP) is an important protocol in Computer Network communications. However, it is also one of the easier protocols to spoof and carry out attacks on, because of the lack of viable solutions to protect the ARP cache. The most common form of attack on the ARP is a man-in-the-middle (MITM) attack, which we will be illustrating in Section 2. Some common programs which users can use to carry out such an attack include `Ettercap' and `Cain and Abel'. 

\section{Background}
In order to understand how ARP spoofing occurs, we will need to cover the basics on how computers communicate with each other. In this section, we will be explaining how computers do so, and introduce the basic notions of how an ARP spoofing is carried out. 

\subsection{Computer Communications}
Firstly, in a typical computer, all IP addresses are resolved dynamically through the use of the Dynamic Host Configuration Protocol (DHCP). This is carried out in the following steps: \\
(to include picture) \\
1. DHCP Discover \\
2. DHCP Offer \\
3. DHCP Request \\
4. DHCP Ack \\ 

After that, suppose the the user Alice wishes to send some information to Bob. The following steps are then carried out: 

(to include picture) \\
1. ARP Request: Alice's computer sends out an ARP request to find out which MAC address has Bob's IP. 

(to include picture) \\
2. When Bob's computer receives the request, he sends back an ARP response packet. 

3. Alice gets the response and stores the corresponding IP-to-MAC entry into the ARP cache. 


\subsection{Poisoning the ARP Cache}
(to include picture) \\
If an attacker, Eve, wishes to carry out an MITM attack on the ARP, this is typically what happens: 

1. Alice's computer sends out an ARP request to find out what is Bob's MAC address. 

2. Before Bob's computer can reply, the attacker, Eve, sends a spam of packets to Alice's computer, claiming to be Bob. The ARP cache then becomes poisoned. 


\section{Current Solutions and \\Mitigations}
There are many solutions in the market to combat ARP Spoofing, such as Agnitum Outpost Firewall. However, from Vivek's\cite{vivek:arp} paper, we can see that many of these solutions employ passive detection, kernal based patches, making MAC entries static, or using a secure ARP protocol. 

\section{Goals}
In our project, we hope to achieve the following goal: \\
1. To provide users with a means to actively combat ARP spoofing \\
2. To make any network more secure. \\
3. Provide a GUI for users to see what is happening in their network in real-time. 

\section{Our Solution}
to insert stuff

\section{Analysis}
to insert stuff 

\section{Limitations and Future Work}
Our project has several flaws which we were unable to resolve within a reasonable timeframe: 

1. Our solution will not work on a network that employs WPA-Enterprise level of encryption. This is because the structure of WPA-Enterprise is such that only each user can see his incoming or outgoing network connections. 

2. Our solution assumes that the user does not have any form of defence installed on his computer. (eg. no firewall that can prevent ARP spoofing, a network that does not use any enterprise level encryption, etc.) 

We hope to improve our solution for a more varied set of systems in the future. 

\subsection{Citations}
Citations to articles \cite{bowman:reasoning, clark:pct, braams:babel, herlihy:methodology},
conference
proceedings \cite{clark:pct} or books \cite{salas:calculus, Lamport:LaTeX} listed
in the Bibliography section of your
article will occur throughout the text of your article.
You should use BibTeX to automatically produce this bibliography;
you simply need to insert one of several citation commands with
a key of the item cited in the proper location in
the \texttt{.tex} file \cite{Lamport:LaTeX}.
The key is a short reference you invent to uniquely
identify each work; in this sample document, the key is
the first author's surname and a
word from the title.  This identifying key is included
with each item in the \texttt{.bib} file for your article.

The details of the construction of the \texttt{.bib} file
are beyond the scope of this sample document, but more
information can be found in the \textit{Author's Guide},
and exhaustive details in the \textit{\LaTeX\ User's
Guide}\cite{Lamport:LaTeX}.

This article shows only the plainest form
of the citation command, using \texttt{{\char'134}cite}.
This is what is stipulated in the SIGS style specifications.
No other citation format is endorsed.

\subsection{Tables}
Because tables cannot be split across pages, the best
placement for them is typically the top of the page
nearest their initial cite.  To
ensure this proper ``floating'' placement of tables, use the
environment \textbf{table} to enclose the table's contents and
the table caption.  The contents of the table itself must go
in the \textbf{tabular} environment, to
be aligned properly in rows and columns, with the desired
horizontal and vertical rules.  Again, detailed instructions
on \textbf{tabular} material
is found in the \textit{\LaTeX\ User's Guide}.

Immediately following this sentence is the point at which
Table 1 is included in the input file; compare the
placement of the table here with the table in the printed
dvi output of this document.

\begin{table}
\centering
\caption{Frequency of Special Characters}
\begin{tabular}{|c|c|l|} \hline
Non-English or Math&Frequency&Comments\\ \hline
\O & 1 in 1,000& For Swedish names\\ \hline
$\pi$ & 1 in 5& Common in math\\ \hline
\$ & 4 in 5 & Used in business\\ \hline
$\Psi^2_1$ & 1 in 40,000& Unexplained usage\\
\hline\end{tabular}
\end{table}

To set a wider table, which takes up the whole width of
the page's live area, use the environment
\textbf{table*} to enclose the table's contents and
the table caption.  As with a single-column table, this wide
table will ``float" to a location deemed more desirable.
Immediately following this sentence is the point at which
Table 2 is included in the input file; again, it is
instructive to compare the placement of the
table here with the table in the printed dvi
output of this document.


\begin{table*}
\centering
\caption{Some Typical Commands}
\begin{tabular}{|c|c|l|} \hline
Command&A Number&Comments\\ \hline
\texttt{{\char'134}alignauthor} & 100& Author alignment\\ \hline
\texttt{{\char'134}numberofauthors}& 200& Author enumeration\\ \hline
\texttt{{\char'134}table}& 300 & For tables\\ \hline
\texttt{{\char'134}table*}& 400& For wider tables\\ \hline\end{tabular}
\end{table*}
% end the environment with {table*}, NOTE not {table}!

\subsection{Figures}
Like tables, figures cannot be split across pages; the
best placement for them
is typically the top or the bottom of the page nearest
their initial cite.  To ensure this proper ``floating'' placement
of figures, use the environment
\textbf{figure} to enclose the figure and its caption.

This sample document contains examples of \textbf{.eps}
and \textbf{.ps} files to be displayable with \LaTeX.  More
details on each of these is found in the \textit{Author's Guide}.

\begin{figure}
\centering
\epsfig{file=fly.eps}
\caption{A sample black and white graphic (.eps format).}
\end{figure}

\begin{figure}
\centering
\epsfig{file=fly.eps, height=1in, width=1in}
\caption{A sample black and white graphic (.eps format)
that has been resized with the \texttt{epsfig} command.}
\end{figure}


As was the case with tables, you may want a figure
that spans two columns.  To do this, and still to
ensure proper ``floating'' placement of tables, use the environment
\textbf{figure*} to enclose the figure and its caption.

Note that either {\textbf{.ps}} or {\textbf{.eps}} formats are
used; use
the \texttt{{\char'134}epsfig} or \texttt{{\char'134}psfig}
commands as appropriate for the different file types.

\begin{figure}
\centering
\psfig{file=rosette.ps, height=1in, width=1in,}
\caption{A sample black and white graphic (.ps format) that has
been resized with the \texttt{psfig} command.}
\end{figure}

\begin{figure*}
\centering
\epsfig{file=flies.eps}
\caption{A sample black and white graphic (.eps format)
that needs to span two columns of text.}
\end{figure*}
 




\section{Conclusions}
ARP spoofing is not easy to correct. etc... 
%\end{document}  % This is where a 'short' article might terminate

%ACKNOWLEDGMENTS are optional
\section{Acknowledgments}
First and foremost, we would like to thank Prof. Hugh Anderson for his guidance and patience with us throughout the semester. While our initial project was to investigate the hacking of Hello Barbie, we eventually decided to change the topic for various reasons, and he very kindly allowed us to do so. 

Our thanks also goes out to him for loaning us the Hello Barbie even though we eventually dropped the project. 

In addition, our appreciation also goes out to our friends who loaned us various items for our project, such as a DLink Dir-825 router which runs on DD-RWT, so that we could work on the project in school. 

%
% The following two commands are all you need in the
% initial runs of your .tex file to
% produce the bibliography for the citations in your paper.
\bibliographystyle{abbrv}
\bibliography{sigproc}  % sigproc.bib is the name of the Bibliography in this case
% You must have a proper ".bib" file
%  and remember to run:
% latex bibtex latex latex
% to resolve all references
%
% ACM needs 'a single self-contained file'!
%
\end{document}
